\documentclass[a4paper,11pt]{article}

\usepackage[T1]{fontenc}
\usepackage[utf8]{inputenc}
\usepackage[italian]{babel}
\usepackage{microtype}
\usepackage{libertine}
\let\mathdollar\undefined
\usepackage{libertinust1math}
\usepackage{MnSymbol}
\usepackage{mathtools}
\usepackage{amsthm}
\usepackage[bb=ams,cal=boondox]{mathalpha}
\renewcommand\ell{\mathcal l}
%\usepackage{amsmath}
%\usepackage{amssymb}
\usepackage{cancel}
\usepackage[dvipsnames]{xcolor}
\usepackage{tikz}
\usepackage{tikz-cd}
\usepackage{pgfplots}
\pgfplotsset{compat=1.18}
\usepackage[theorems,skins,breakable]{tcolorbox}
\usepackage{import}
\usepackage{pdfpages}
\usepackage{transparent}
\usepackage{enumitem}
\usepackage[colorlinks]{hyperref}

\newcommand*{\sminus}{\raisebox{1.3pt}{$\smallsetminus$}}

\newcommand*{\transp}[2][-3mu]{\ensuremath{\mskip1mu\prescript{\smash{\mathrm t\mkern#1}}{}{\mathstrut#2}}}%

% newcommand for span with langle and rangle around
\newcommand{\Span}[1]{{\left\langle#1\right\rangle}}

\newcommand{\incfig}[2][1]{%
    \def\svgwidth{#1\columnwidth}
    \import{./figures/}{#2.pdf_tex}
}

\newcounter{theo}[section]
\counterwithin{theo}{section}

\newcounter{excounter}[section]
\counterwithin{excounter}{section}

\counterwithin{equation}{section}

\newtcolorbox{lemmao}[1][]{
    breakable, enhanced jigsaw, left=2pt, right=2pt, top=4pt, bottom=2pt, 
    colback=white, colframe=green!20,
    title=Lemma~\stepcounter{theo}\thetheo\ifstrempty{#1}{}{:~#1},
    coltitle=black, fonttitle=\bfseries,
    attach boxed title to top left={xshift=3mm,yshift=-2mm},
    boxed title style={colback=green!20}
}

\newtcolorbox{theorem}[1][]{
    breakable, enhanced jigsaw, left=2pt, right=2pt, top=4pt, bottom=2pt, 
    colback=white, colframe=blue!20,
    title=Teorema~\stepcounter{theo}\thetheo\ifstrempty{#1}{}{:~#1},
    coltitle=black, fonttitle=\bfseries,
    attach boxed title to top left={xshift=3mm,yshift=-2mm},
    boxed title style={colback=blue!20}
}

\newtcolorbox{definition}[1][]{
    breakable, enhanced jigsaw, left=2pt, right=2pt, top=4pt, bottom=2pt, 
    colback=white, colframe=violet!20,
    title=Definizione~\stepcounter{theo}\thetheo\ifstrempty{#1}{}{:~#1},
    coltitle=black, fonttitle=\bfseries,
    attach boxed title to top left={xshift=3mm,yshift=-2mm},
    boxed title style={colback=violet!20}
}

\theoremstyle{definition}
\newtheorem{lemma}[theo]{Lemma}
\newtheorem{corollary}[theo]{Corollario}
\newtheorem{proposition}[theo]{Proposizione}

%\theoremstyle{definition}
\newtheorem{example}[theo]{Esempio}

%\theoremstyle{remark}
\newtheorem*{note}{Nota}
\newtheorem*{remark}{Osservazione}

\newtcolorbox{notebox}{
  colback=gray!10,
  colframe=black,
  arc=5pt,
  boxrule=1pt,
  left=15pt,
  right=15pt,
  top=15pt,
  bottom=15pt,
}

\DeclareRobustCommand{\rchi}{{\mathpalette\irchi\relax}} % beautiful chi
\newcommand{\irchi}[2]{\raisebox{\depth}{$#1\chi$}} % inner command, used by \rchi

\newtcolorbox{eser}[1][]{
    colframe=black!0,
    coltitle=black!70, % Title text color
    fonttitle=\bfseries\sffamily, % Title font
    title={Esercizio~\stepcounter{theo}\thetheo~#1}, % Title format
    sharp corners, % Less rounded corners
    boxrule=0pt, % Line width of the box frame
    toptitle=1mm, % Distance from top to title
    bottomtitle=1mm, % Distance from title to box content
    colbacktitle=black!5, % Background color of the title bar
    left=0mm, right=0mm, top=1mm, bottom=1mm, % Padding around content
    enhanced, % Enable advanced options
    before skip=10pt, % Space before the box
    after skip=10pt, % Space after the box
    breakable, % Allow box to split across pages
    colback=black!0,
    borderline west={1pt}{-5pt}{black!70},
}
\newcommand{\seminorm}[1]{\left\lvert\hspace{-1 pt}\left\lvert\hspace{-1 pt}\left\lvert#1\right\lvert\hspace{-1 pt}\right\lvert\hspace{-1 pt}\right\lvert}

%%% Local Variables:
%%% mode: LaTeX
%%% TeX-master: "Analisi_Complessa/Analisi_complessa"
%%% End:
